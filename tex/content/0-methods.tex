% !TEX root = ../thesis-example.tex
%
\section*{Overview}
%\vspace*{-5mm}
\label{sec:methods}

Title: Personal Perception and Interaction with Medical Information

\begin{enumerate}
	\item
		\emph{Introduction}: The history of MR, focusing on how to improve the perception of the digital information and the corresponding interactive methods. objectives and requirements in different stages and how the technology develops.
		Personal perception =? Magic Mirror
	\item	
		\emph{The state of art} The current states: some famous MR system, the main contributions of the MRs and how to improve Personal Perception and Interaction. 
	
	\item
		\emph{Personal Perception with Magic Mirror}: How to improve the perception in the Magic Mirror framework.
		\begin{enumerate}
		\item	\emph{Magic Mirror framework} conception, software and hardware
		\item	\emph{Accuracy of Registration} improve the perception of the MR view via accurate registration. Anatomy learning, personal information (gender, age, body shape)
		\item	\emph{Interactive MR} Make the user believe the virtual element is a part of his or her own body. (Organ game and the Muscle learning)
		\end{enumerate}
	
	\item
		\emph{Personal Pointing Interaction}: The interaction with MR via pointing gesture
		\begin{enumerate}
			\item \emph{Calibration and Recovery of Pointing gesture} The PAST method and the evaluation. 
			\item \emph{Interaction with Multi-screen in OR} 
			%\item \emph{Augmented Human to access the real world} The case without depth info. Natural pointing interaction for optical see-through AR to help wearable computer to fetch the information%for UbiComp
		\end{enumerate}
	\item \emph{Collaborative Mixed Reality} combine the technology for perception and interaction (Maybe Collaborative User interface or platform. )
	%todo: a survey about how to evaluate a collaborative AR system.
		\begin{enumerate}
			\item \emph{Magic mirror with pointing gesture interaction} One patient <-> one Kinect and one display  doing the rehabilitation exercise.
			One nurse with pointing gesture device: to control all the exercises. 
			Parents monitor children’s magic mirror  teacher monitor student’s magic mirror 
			\item \emph{Shared the view} one user perform pointing, another user really see the pointing gesture in his or her own view. to collaborate with each other. For teaching and education. 
		\end{enumerate}
\end{enumerate}




