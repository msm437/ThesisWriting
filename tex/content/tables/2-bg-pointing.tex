% !TEX root = ../thesis-example.tex
%\newgeometry{margin=1cm} % modify this if you need even more space
%\begin{landscape}
\begin{sidewaystable}
	\caption{The objects to track and the properties of the interaction in different works}
	\label{tb:2-bg:pointing}
	\scriptsize
	\centering
	\begin{threeparttable}
		\begin{tabular}{P{5cm}|P{1cm}|P{1cm}|P{1cm}|P{1cm}|P{1cm}|P{1cm}|P{2cm}|P{2cm}|P{2cm}|P{2cm}}
			\hline
			\space & \multicolumn{4}{c}{Objects to Track} & \multicolumn{6}{|c}{Interaction} \\
			\hline
			Work & Head & Eye & Hand & Finger & Visual Feedback & Direct & Dynamic Scene & Unlimited Distance & Real World & Virtual World\\
			%\hline
			\citet{Pierce1997} & \xmark & \xmark \tnote{a}  & \cmark \tnote{b} & \cmark \tnote{b} & \cmark & \cmark & \cmark & \cmark & \xmark & \cmark \\
			\citet{Argelaguet2008} & \xmark & \xmark \tnote{a}  & \cmark & \cmark & \cmark & \cmark & \cmark & \cmark & \xmark & \cmark \\
			\citet{Liang1994} & \xmark & \xmark \tnote{a}  & \cmark\tnote{b} & \cmark\tnote{b} & \cmark & \cmark & \xmark & \cmark & \cmark & \cmark \tnote{c} \\
			\citet{Nickel2003} & \cmark & \cmark  & \cmark & \cmark & \cmark & \cmark & \xmark & \cmark & \cmark & \xmark\\
			\citet{Banerjee2012} & \xmark & \cmark  & \cmark & \cmark & \cmark & \cmark & \xmark & \cmark & \cmark & \xmark\\
			\citet{Huang2014} & \xmark & \cmark  & \xmark & \cmark & \xmark & \cmark & \xmark & \cmark & \cmark & \xmark\\
			\citet{DeCampos2006} & \xmark & \xmark  & \cmark & \cmark & \cmark & \xmark & \cmark & \cmark & \cmark \tnote{d} & \cmark \tnote{d}\\
			\citet{Colaco2013a} & \xmark & \xmark  & \cmark & \cmark & \cmark & \xmark & \cmark & \cmark & \cmark & \cmark\\
			\citet{Mistry2009} & \xmark & \xmark  & \xmark & \cmark & \cmark & \xmark & \cmark & \cmark & \cmark & \xmark\\
			\citet{Harrison2011} & \xmark & \xmark  & \cmark & \cmark & \xmark & \cmark & \cmark & \xmark & \cmark & \xmark\\
			\citet{Ha2014} & \xmark & \xmark \tnote{a}  & \cmark & \cmark & \cmark & \cmark & \cmark & \cmark & \cmark \tnote{e} & \cmark\\
			\citet{Jang2015} & \xmark & \xmark \tnote{a}  & \cmark & \cmark & \cmark & \cmark & \cmark & \xmark & \xmark & \cmark\\
			\citet{Kassner2014} & \xmark & \cmark  & \xmark & \xmark & \cmark & \cmark & \cmark & \cmark & \cmark \tnote{d} & \cmark \tnote{d}\\
			Our method &  \xmark & \xmark \tnote{f} & \xmark & \cmark & \xmark & \cmark & \cmark & \cmark & \cmark \tnote{d} & \cmark \tnote{d}\\
			\end{tabular}
			\begin{tablenotes}
				\item[a] The eye position is defined during rendering procedure.
				\item[b] Special tool is tracked instead of hand and finger.
				\item[c] 3D virtual world is shown in 2D display.
				\item[d] The method could be used in this scenario with proper hardware.
				\item[e] The Real world is shown via video see through HMD.
				\item[f] The eye position is calculated via the tracking of the finger and scenario.
				\end{tablenotes}
				\end{threeparttable}
				\end{sidewaystable}
%\end{landscape}
%\restoregeometry