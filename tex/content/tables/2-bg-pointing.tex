% !TEX root = ../thesis-example.tex
%\newgeometry{margin=1cm} % modify this if you need even more space
%\begin{landscape}
%\begin{table} %sidewaystable
\begin{table*}[t]
	\caption{The objects to track and the properties of the interaction in different works}
	\label{tb:relatedWork}
	\scriptsize
	\centering
	\begin{threeparttable}
		\centering
		\begin{tabular}{P{1.1cm}|P{0.35cm}|P{0.25cm}|P{0.35cm}|P{0.45cm}|P{1.0cm}|P{1.2cm}|P{0.8cm}|P{1.2cm}|P{0.5cm}|P{0.5cm}}
			\hline
			\space & \multicolumn{4}{c}{Objects to Track} & \multicolumn{6}{|c}{Properties of the Interaction} \\
			\hline
			Work & Head & Eye & Hand & Finger & Visual Feedback & Direct Positioning & Dynamic Scene & Unreachable Objects & Real World & Virtual World\\
			%\hline
			\citep{Pierce1997} & \xmark & \xmark \tnote{a}  & \cmark \tnote{b} & \cmark \tnote{b} & \cmark & \cmark & \cmark & \cmark & \xmark & \cmark \\
			\citep{Argelaguet2008} & \xmark & \xmark \tnote{a}  & \cmark & \cmark & \cmark & \cmark & \cmark & \cmark & \xmark & \cmark \\
			\citep{Liang1994} & \xmark & \xmark \tnote{a}  & \cmark\tnote{b} & \cmark\tnote{b} & \cmark & \cmark & \xmark & \cmark & \cmark & \cmark \tnote{c} \\
			\citep{Nickel2003} & \cmark & \cmark  & \cmark & \cmark & \cmark & \cmark & \xmark & \cmark & \cmark & \xmark\\
			\citep{Banerjee2012} & \xmark & \cmark  & \cmark & \cmark & \cmark & \cmark & \xmark & \cmark & \cmark & \xmark\\
			\citep{Huang2014} & \xmark & \cmark  & \xmark & \cmark & \xmark & \cmark & \xmark & \cmark & \cmark & \xmark\\
			\citep{DeCampos2006} & \xmark & \xmark  & \cmark & \cmark & \cmark & \xmark & \cmark & \cmark & \cmark \tnote{d} & \cmark \tnote{d}\\
			\citep{Colaco2013a} & \xmark & \xmark  & \cmark & \cmark & \cmark & \xmark & \cmark & \cmark & \cmark & \cmark\\
			\citep{Mistry2009} & \xmark & \xmark  & \xmark & \cmark & \cmark & \xmark & \cmark & \cmark & \cmark & \xmark\\
			\citep{Harrison2011} & \xmark & \xmark  & \cmark & \cmark & \xmark & \cmark & \cmark & \xmark & \cmark & \xmark\\
			\citep{Ha2014} & \xmark & \xmark \tnote{a}  & \cmark & \cmark & \cmark & \cmark & \cmark & \cmark & \cmark \tnote{e} & \cmark\\
			\citep{Jang2015} & \xmark & \xmark \tnote{a}  & \cmark & \cmark & \cmark & \cmark & \cmark & \xmark & \xmark & \cmark\\
			\citep{Kassner2014} & \xmark & \cmark  & \xmark & \xmark & \cmark & \cmark & \cmark & \cmark & \cmark \tnote{d} & \cmark \tnote{d}\\
			\\
			%Our method &  \xmark & \xmark \tnote{f} & \xmark & \cmark & \xmark & \cmark & \cmark & \cmark & \cmark \tnote{d} & \cmark \tnote{d}\\
		\end{tabular}
		\begin{tablenotes}
			\item[a] The eye position is defined during the rendering procedure.
			\item[b] Special tool is tracked instead of hand and finger.
			\item[c] 3D virtual world is shown in 2D display.
			\item[d] The method could be used in this scenario with proper hardware.
			\item[e] The real world is shown via video-see-through HMD.
			\item[f] The eye position is calculated via the tracking of the finger and scenario.
		\end{tablenotes}
	\end{threeparttable}
\end{table*}
%\end{landscape}
%\restoregeometry