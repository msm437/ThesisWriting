% !TEX root = Clean-Thesis.tex
%
\pdfbookmark[0]{Abstract}{Abstract}
\chapter*{Abstract}
\label{sec:abstract}
%\vspace*{-10mm}

%computer education, Mixed Reality, medical image, interaction, and perception. 

%Personal Perception and Interaction with Medical information in magic mirror framework

At first, techniques for improved 2D B-mode ultrasound visualization are presented, where the observer is provided with real-time feedback on the uncertainty present in the image.
Therefore, in a first step, the necessary information is generated through the computation of ultrasound Confidence Maps estimating per-pixel uncertainty information with respect to the ultrasound signal attenuation.
While their original formulation has been proven to improve tasks such as image registration and segmentation, their valuable information has never been directly exposed to the clinician.
Therefore, a real-time capable extension in form of an incremental solver scheme is presented and shown to yield precise results for visualization applications.
The uncertainty information is eventually fused with the original B-mode images using three carefully chosen visualization schemes.
Their benefit has been shown for both educational and clinical applications.

%pointing gesture recovery in egocentric setting

%natural interaction and switching between multi-system in Operating room
The second part of this thesis' contributions targets improved 3D ultrasound visualization.
In order to generate high-quality 3D volumes from tracked ultrasound sweeps, first an incremental compounding scheme is presented, which combines orientation-driven correlation terms and the aforementioned Confidence Maps in an information fusion approach.
The subsequent visualization of such ultrasound volumes is a particularly challenging task due to the special nature of B-mode intensities.
Therefore, a novel classification concept in the form of point predicates is presented, which seamlessly integrates into the standard direct volume rendering pipeline.

%natrual perception and one-multi user supervision and teaching, training 
In conjunction with the proposed predicate histogram as intuitive user interface, this technique allows for much more meaningful visualization and facilitates the understanding of the data for the observer.

%\vspace*{20mm}
%
%{\usekomafont{chapter}Zusammenfassung}\label{sec:abstract-diff} \\
%
%\blindtext
