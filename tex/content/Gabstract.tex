% !TEX root = Clean-Thesis.tex
%
\pdfbookmark[0]{Zusammenfassung}{Zusammenfassung}
\chapter*{Zusammenfassung}
\label{sec:Zusammenfassung}
\vspace*{-15mm}
Medizinische Informationen werden bei Ausbildung, Training, Diagnose und Operationen benutzt. Die Wahrnehmung von unterschiedlichen Medien und Gegenständen und die Interaktion damit sind grundlegende menschliche Aktivitäten und sie sind benutzerspezifisch. Wissenschaftler arbeiten an Systemen, welche die Darstellung und Interaktion mit medizinischen Informationen den Bedürfnissen von Studenten, Patienten, und Doktoren anpassen. 

Zuerst wird der magische Spiegel vorgestellt, der reale Bilder mit digitalen Informationen überlagert. Dazu verfolgt das System die Pose des Benutzers in Echtzeit und passt mit Hilfe eines Kalibrierungsalgorithmus die Visualisierung von medizinischen Datensätzen spezifisch auf die Proportionen des jeweiligen Benutzers an. Dies bietet interessante Möglichkeiten für benutzerspezifische erweiterte Realität in der medizinischen Ausbildung und für Rehabilitation. In zwei Benutzerstudien wurde die Genauigkeit der überlagerten Informationen und die Akzeptanz des Systems für Anatomie Ausbildung überprüft. Das System wurde verwendet um Benutzern die menschlichen Organe und Muskeln zu lehren, sowie zur Rehabilitation. Zusätzliche Funktionen wie zum Beispiel ``Organ Explosion'' und ``Selbst-gesteuerte virtuelle Ansicht'' wurden entworfen und implementiert. Ernsthafte Spiele wurden auch auf Basis des magischen Spiegels entwickelt.

Eine neuartige Mensch-Maschine Schnittstelle mithilfe einer tragbaren Tiefenkamera wurde ebenfalls entwickelt. Sie ermöglicht es Benutzern durch einfaches zeigen auf Gegenstände mit dem System zu interagieren. Dazu schlagen wir eine neuartige Kalibrierungsmethode vor, welche die Position des Blickpunkts eines Benutzers ermittelt. Ziele werden durch eine gerade Linie anvisiert, welche definiert ist durch den Blickpunkt und den Zeigefinger des Benutzers. Diese intuitive Art der Interaktion mit Gegenständen in der Umgebung erlaubt die Kontrolle von Programmen ohne weitere visuelle Rückmeldung. Bei einer mathematischen Analyse der Kalibrierungsmethode konnten wir zeigen, dass der Fehler der Zielstrahl Berechnung unter $0,9\degree$ liegt. Mithilfe einer Benutzerstudie haben wir die Genauigkeit der vorgestellten Kalibrierungsmethode gemessen. Benutzer konnten die vorgegebenen Ziele mit einer Genauigkeit von $0,67\degree \pm 0,71\degree$ ohne visuelle Rückmeldung erreichen. Die intuitive Interaktionsmethode erlaubt es Ärzten eigenständig mit diversen medizinischen Geräten berührungslos zu interagieren. Die Implementierung des Benutzerinterfaces ermöglicht zudem eine Integration in bestehende Systeme ohne Änderungen an Geräten und den benutzten Programmen.

Zum Schluss wird der magische Spiegel mit der berührungslosen Interaktion kombiniert und ein System vorgestellt, das von mehreren Benutzern gleichzeitig bedient werden kann. Die zwischenmenschliche Kommunikation wird durch die Unterstützung von Zeigegesten verbessert.

\textbf{Schlagwörter:} Medizinische Informationen, personalisierte Wahrnehmung, Personalisierte Interaktion, Augmented Reality, Zeigegeste, Mixed-Reality

%The framework is simply examined in two scenarios, anatomy teaching with a student and rehabilitation exercise with a nurse.

%\vspace*{20mm}
%
%{\usekomafont{chapter}Zusammenfassung}\label{sec:abstract-diff} \\
%
%\blindtext
