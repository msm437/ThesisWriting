% !TEX root = ../thesis-example.tex

%\chapter{Personalized Perception in the Magic Mirror Framework} \label{chaptor:3}
\chapter{Personalized Augmented Reality} \label{chaptor:3}
%why we develop a magic mirror framework: the advantage of magic mirror: to create a personal perception of the medical information
Providing adequate learning experience to different learners is a challenging issue as the traditional learning system generally can not adapt content to suit individual learner needs. Personalization for promoting a multi-modal learning environment is a growing area of interest, such as the development of user modeling and personalized processes which place the student at the center of the learning development.
AR systems can present a virtual representation of the subject matter and create a direct connection between the information the user wants to learn and his own body and activities at the same time. Hence, it could help understand and memorize complex information, and either supplement conventional learning or even supersede it altogether. 
Previous AR/MR systems on visualization of anatomy used expensive systems involving HMDs or markers. The system presented in this chapter is an inexpensive and easy to use AR system, which takes advantage of the magic mirror concept to present medical information. It presents medical data augmented onto the user's body and shows additional 2D and 3D information according to the need. The magic mirror concept provide the user `the superman ability' to look into his/her own body. It enables the medical information to be perceived naturally linking to a real human body. Natural gesture is chosen as the interaction methodology, and interaction with the AR view of the user's own body provides a personalized perception in the magic mirror framework.

The magic mirror framework with medical information is firstly presented in Section \ref{sec:3-PPMM:MMC}, including the hardware setup, the system framework and some important system features. An example application for anatomy education is created using the proposed framework, and a survey with 72 fresh medical students was done to evaluate the conception and find the direction to improve the framework. In Section \ref{sec:3-PPMM:Registration}, the user specific information is collected to improve the registration of the AR view and enhance the personalized perception. Section \ref{sec:3-PPMM:IMR} takes the advantage of the interactive mixed reality to generate a personalized learning procedure, and systems are developed and evaluated for anatomy leaning, especially for muscle learning. At last some serious games are developed in the magic mirror framework for public education and rehabilitation.