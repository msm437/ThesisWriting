%todo: Make the user believe the virtual element is a part of his or her own body. (Organ game and the Muscle learning)
\section{Interactive Mixed Reality}


\subsection{Personalized natural interactin \& gaming}

\paragraph{Gesture-based interaction}: Medical volumes are usually visualized by showing slices that are aligned with the axes of the volume. A volume can be seen as a stack of transverse slices starting from the top and going to the bottom. When the system is in the transverse slice mode, the user could move their hand up or down to choose the interesting slice image, respectively left or right for the sagittal slice mode. The current slice is depicted on the right part of the monitor while a yellow circle is augmented onto the user depicting the slice plane. The system can easily switch slices between the CT and the photographic volume.

\paragraph{Organ explosion effect}: In the AR learning environment, organ selection is a challenging task as some organs hide behind others. The user can’t directly position their hand into their bodies and to select the organ of choice. In engineering design, mechanical parts are drawn with the innermost part at the center, while the others are moved some fixed distance outwards to display all parts of the assembly that would otherwise be hidden. Inspired by this principle we designed and developed an organ explosion effect (see Figure 3-left), which separates the organs. The organ explosion selection employs a two-hand interaction method. Using the left hand, the user is able to focus the height for the section they are interested in. Organs at approximately the same height are then projected outwards. Using a spherical projection, the organs are moved outward, creating the illusion of seeing them ‘fly out’ in front of the body. Lines are drawn to indicate the original position of the organs. After separating the organs from each other, it is easy for the user to select the organ of interest using their right hand. Another functionality of the organ explosion effect is allowing the user to rotate and observe the 3D organ models and perceive the spatial relationship between them.

\paragraph{Serious gaming}: Health education is a particular area where serious gaming is suited \cite{Aubin2012,Jonas2012}. We developed a game for general users to learn basic anatomy called Whack an Organ (see Figure 3-right). Our game is based on the idea of the classic arcade games at carnivals, ‘Whack-a-Mole’. We implement a similar game idea using our system framework. It is a combination of Whack-a-Mole and classical quiz games: the user is presented with questions regarding human anatomy, and the answers are always organs or organ systems. To answer the question, the user has to point to the location of the organ directly on their body. The game then decides if the location is correct and awards points if so. Questions can come from different question sets with varying difficulty, ranging from simple location questions: where is the liver?, to more complex knowledge questions: which organ is infected if you have Hepatitis? 